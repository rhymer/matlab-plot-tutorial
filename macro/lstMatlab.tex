% Matlab listing macros.
%   First we define the Matlab listing style. There are two classoffset
%   beside the basic keywords. The first offset is for command style 
%   code with purple color. The second offset is the emphisized keywords.
%   Feel free to edit or remove them.
%
%   We provide two commands to use the Matlab code listing style.
%
%   The matlabcode environment:
%       \begin{matlabcode}[additional_styles]
%           put your Matlab code here
%       \end{matlabcode}
%
%   The input method:
%       \inputMatlabCode{filename}[additional_style]

\definecolor{dkgreen}{rgb}{0,0.6,0}  % Comment.
\definecolor{purple}{rgb}{0.6,0,0.6} % String.
\lstloadlanguages{Matlab}
\lstdefinestyle{Matlab}  % Define the Matlab listing style.
{
   language=Matlab,
   basicstyle=\ttfamily,
   numbers=none, % {right, left, none}
   numberstyle=\tiny\color{black},
   stepnumber=1,
   numbersep=10pt,
   backgroundcolor=\color{white},
   tabsize=4,
   showspaces=false,
   showstringspaces=false,
   classoffset=0, %=================================================== 
   frame=single,  % Single frame around code
   keywords={break,case,catch,continue,else,elseif,end,for,function,
             global,if,otherwise,persistent,return,switch,try,while},
   keywordstyle=\color{blue},
   commentstyle=\usefont{T1}{pcr}{m}{sl}\color{dkgreen},
   stringstyle=\color{purple},
   classoffset=1, %===================================================
   morekeywords={all,on,off},keywordstyle=\color{purple},
   classoffset=2, %===================================================
   morekeywords={},
   keywordstyle=\color{black}\bfseries,
   classoffset=0  %===================================================
}

\lstnewenvironment{matlabcode}[1][]{\lstset{style=Matlab,#1}}{}
\newcommand{\inputMatlabCode}[2][]
           {\lstinputlisting[style=Matlab,title=\lstname,#1]{#2}}